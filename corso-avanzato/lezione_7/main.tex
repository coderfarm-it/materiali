\documentclass[compress]{beamer}
\usepackage[utf8]{inputenc}
\usepackage[english]{babel}
\usepackage{hyperref}
\usepackage{ccicons}

\usepackage{tikz}
\usetikzlibrary{graphs, quotes, arrows.meta, matrix}

\usetheme{default}
\usecolortheme{Nord}
\setbeamertemplate{navigation symbols}{}

\title{Esercizi sui grafi}
\subtitle{Dijkstra, DSU e DP su grafi}
\author{Lorenzo Ferrari, Davide Bartoli}
\date{\today}

\begin{document}

\begin{frame}
  \maketitle
\end{frame}

\begin{frame}{Table of contents}
  \tableofcontents
\end{frame}

\section{Problemi}

\subsection{investigation}
\begin{frame}{Problemi}{investigation}
    \underline{\url{https://cses.fi/problemset/task/1202}}
    \begin{itemize}
        \item ci vengono chieste diverse informazioni riguardo gli shortest path
        \pause
    \item dobbiamo fare contemporaneamente dijkstra e dp su un grafo
    \end{itemize}
\end{frame}

\subsection{ois\_xmastree}
\begin{frame}{Problemi}{ois\_xmastree}
    \underline{\url{https://training.olinfo.it/\#/task/ois_xmastree/statement}}
    \begin{itemize}
        \item sembra tanto una dp su albero
        \pause
        \item radichiamo l'albero in $0$
        \pause
        \item se switcho un nodo diverso dalla root, anche il suo parent viene switchato
        \item per ogni nodo  $v$ non foglia, il numero di volte che $v$ viene switchato \`e la somma del numero di volte in cui i figli vengono switchati
        \pause
        \item siano \texttt{dp[v][0]} e \texttt{dp[v][1]} la risposta per il subtree del nodo $v$, dove $v$ \`e switchato rispettivamente un numero pari e un numero dispari di volte
    \end{itemize}
\end{frame}

\subsection{ois\_patrol}
\begin{frame}{Problemi}{ois\_patrol}
    \underline{\url{https://training.olinfo.it/\#/task/ois_patrol/statement}}
\end{frame}


\subsection{ois\_cannons}
\begin{frame}{Problemi}{ois\_cannons}
    \underline{\url{https://training.olinfo.it/\#/task/ois_cannons/statement}}
    \begin{itemize}
        \item prima idea importante: vediamo il problema come un grafo pesato, gli archi in input hanno costo 0, 
        gli archi che vanno dal nodo $i$ al nodo $j$ hanno costo $j-i$
        \pause
        \item possiamo quindi fare dijkstra per trovare la risposta al problema. Il numero di archi però è $O(N^2)$, 
        quindi questa soluzione non ottiene punteggio pieno.
    \end{itemize}
    \pause
    Possiamo ridurre il numero di archi senza perdere informazioni?
    \pause
    Rappresentando gli archi aggiunti da noi, notiamo che alcuni non sono necessari!
\end{frame}

\subsection{crocodile}
\begin{frame}{Problemi}{crocodile}
    \underline{\url{https://training.olinfo.it/\#/task/crocodile/statement}}
    \pause
    \begin{itemize}
        \item il problema non \`e banale
        \item facciamo dijkstra al contrario, partendo dalle uscite
        \pause
        \item per ogni nodo, teniamo le due distanze minime
        \item ``propaghiamo'' con la seconda distanza minima: l'avversario bloccher\`a il cammino migliore
    \end{itemize}

    \pause
    Un problema molto simile \`e \texttt{pre\_boi\_sbarramento}

    \small{\underline{\url{https://training.olinfo.it/\#/task/pre_boi_sbarramento/statement}}}
\end{frame}

\begin{frame}{Problemi}
    \underline{\url{https://cses.fi/problemset/task/1202}}
    \underline{\url{https://cses.fi/problemset/task/1679}}
    \underline{\url{https://training.olinfo.it/\#/task/ois_patrol/statement}}
    \underline{\url{https://training.olinfo.it/\#/task/ois_xmastree/statement}}
    \underline{\url{https://training.olinfo.it/\#/task/ois_words/statement}}
    \underline{\url{https://training.olinfo.it/\#/task/ois_cannons/statement}}
    \underline{\url{https://training.olinfo.it/\#/task/ois_minperm/statement}}
    \underline{\url{https://training.olinfo.it/\#/task/ois_waterslide/statement}}
    \underline{\url{https://training.olinfo.it/\#/task/crocodile/statement}}
\end{frame}

\end{document}
